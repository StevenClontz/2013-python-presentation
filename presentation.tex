% $Header: /home/vedranm/bitbucket/beamer/solutions/generic-talks/generic-ornate-15min-45min.en.tex,v 90e850259b8b 2007/01/28 20:48:30 tantau $

\documentclass{beamer}

% This file is a solution template for:

% - Giving a talk on some subject.
% - The talk is between 15min and 45min long.
% - Style is ornate.



% Copyright 2004 by Till Tantau <tantau@users.sourceforge.net>.
%
% In principle, this file can be redistributed and/or modified under
% the terms of the GNU Public License, version 2.
%
% However, this file is supposed to be a template to be modified
% for your own needs. For this reason, if you use this file as a
% template and not specifically distribute it as part of a another
% package/program, I grant the extra permission to freely copy and
% modify this file as you see fit and even to delete this copyright
% notice. 


\mode<presentation>
{
  \usetheme{Warsaw}
  % or ...

  \setbeamercovered{transparent}
  % or whatever (possibly just delete it)
}


\usepackage[english]{babel}
% or whatever

\usepackage[latin1]{inputenc}
% or whatever

\usepackage{times}
\usepackage[T1]{fontenc}
% Or whatever. Note that the encoding and the font should match. If T1
% does not look nice, try deleting the line with the fontenc.


\usepackage{marvosym} % For \Smiley
\usepackage{verbatim} % for \verbatiminput

\title[Quick Intro to Python] % (optional, use only with long paper titles)
{Quick Intro to Python}

% \subtitle
% {} % (optional)

\author%[Author, Another] % (optional, use only with lots of authors)
{Steven~Clontz}%\inst{1} \and S.~Another\inst{2}}
% - Use the \inst{?} command only if the authors have different
%   affiliation.

\institute[Auburn University] % (optional, but mostly needed)
{
  %\inst{1}%
  COMP 7970\\
  Auburn University}
  %\and
  %\inst{2}%
  %Department of Theoretical Philosophy\\
  %University of Elsewhere}
% - Use the \inst command only if there are several affiliations.
% - Keep it simple, no one is interested in your street address.

\date[13-02-??] % (optional)
{February ??, 2013}

\subject{Quick Intro to Python}
% This is only inserted into the PDF information catalog. Can be left
% out. 



% If you have a file called "university-logo-filename.xxx", where xxx
% is a graphic format that can be processed by latex or pdflatex,
% resp., then you can add a logo as follows:

 \pgfdeclareimage[height=1cm]{university-logo}{auburn_logo.png}
 \logo{\pgfuseimage{university-logo}}



% Delete this, if you do not want the table of contents to pop up at
% the beginning of each subsection:
%\AtBeginSubsection[]
%{
%  \begin{frame}<beamer>{Outline}
%    \tableofcontents[currentsection,currentsubsection]
%  \end{frame}
%}


% If you wish to uncover everything in a step-wise fashion, uncomment
% the following command: 

%\beamerdefaultoverlayspecification{<+->}


\begin{document}

\begin{frame}
  \titlepage
\end{frame}

\begin{frame}{Table of Contents}
  \tableofcontents
  % You might wish to add the option [pausesections]
\end{frame}


% Since this a solution template for a generic talk, very little can
% be said about how it should be structured. However, the talk length
% of between 15min and 45min and the theme suggest that you stick to
% the following rules:  

% - Exactly two or three sections (other than the summary).
% - At *most* three subsections per section.
% - Talk about 30s to 2min per frame. So there should be between about
%   15 and 30 frames, all told.

\section{Introduction}

\begin{frame}{Abstract}%{Subtitles are optional.}
  % - A title should summarize the slide in an understandable fashion
  %   for anyone how does not follow everything on the slide itself.

  \begin{itemize}
  \item
    Presentation available at https://github.com/StevenClontz/2013-python-presentation
  \pause
  \item
    I'll talk about basic syntax, data types, loops, functions, conditionals etc. in Python.
    \pause
  \item
    It should be enough to get anyone started hacking, at least.
  \end{itemize}
\end{frame}

\section{Getting Started}

\begin{frame}{Getting Started}

  \begin{itemize}
  \item
    Python's syntax encourages beautiful code, using whitespace to organize rather than combinations of braces and semicolons.
  \pause
  \item
    We also don't worry about declaring variables or defining complex data types - most of what you'll need comes in the box.
  \pause
    If it's not there, we can always import it.
  \end{itemize}
\end{frame}

\begin{frame}[fragile]
\textbf{C and Python:}
\begin{verbatim}
void foo(int x)        def foo(x):
{if (x == 0) {           if x == 0:
                           bar()
    bar();baz();           baz()
                         else:
  } else {                 qux(x)
    qux(x);                foo(x - 1)
    foo(x - 1);}
}
\end{verbatim}
\end{frame}

% \begin{frame}[fragile]
% \textbf{Python Code:}

% \verbatiminput{code/syntax_example.py}
% \end{frame}

% \begin{frame}[fragile]
% \textbf{Output:}

% \begin{verbatim}
% Let's say hi to everyone five times
% Hello world! (x1)
% Hello world! (x2)
% Hello world! (x3)
% Hello world! (x4)
% Hello world! (x5)
% Here's a random decimal number between 0 and...
% 32.2
% \end{verbatim}
% \end{frame}

\section{Data Types}

\begin{frame}[fragile]{Data Types}
  \begin{itemize}
  \item
    Strings:

    \verb\name = "Stevie"\

    \verb\hello = 'Hi %s' % name  # or 'Hi $name'\

    \verb\blank = ""  # or str()\

  \pause
  \item
    Numbers:

    \verb\three = 1 + 2  # an integer\

    \verb\one_point_two_ish = -1 + 2.2  # a float\

    \verb\degree_135 = -2 + 2j  # gaussian integer\

  \pause
    \begin{itemize}
      \item
        Big numbers? No problem!

        \verb\small = 3\

        \verb\big = small**small**small\

        \verb\  # 3^(3^3) = 7625597484987\
    \end{itemize}
  \end{itemize}
\end{frame}

\begin{frame}[fragile]{Data Types}
  Iterables:
  \pause
  \begin{itemize}
  \item
    Lists:

    \verb\less_than_four = [0, 1, 2, 3]  # or range(4) \

    \verb\two_words = list().append("Hi").append("Mom")\

    \verb\comprende = [2**n for n in less_than_four]\

    \verb\  # [1, 2, 4, 8]\
  \pause
  \item
    Tuples:

    \verb\five_away = (3, 4)\

    \verb\five_away.append(1) # ERROR!\

  \end{itemize}
\end{frame}

\begin{frame}[fragile]{Data Types}
  More iterables:

  \begin{itemize}
  \item
    Dictionaries:

\begin{verbatim}
named = {
  "one": 1,
  "two comma three": (2, 3),
  4: "four (backwards)"
}
\end{verbatim}

  \verb\  # named['one'] == 1\

  \verb\emptydict = dict()  # or {}\
  \pause
  \item
    Sets:

    \verb\no_dupes = set([1, 2, 1, "two"])\

    \verb\  # or {1, 2, "two"}\

    \verb\emptyset = set()  # not {}\

  \end{itemize}
\end{frame}

\section{Operators}

\begin{frame}[fragile]{Operators}
 
    \verb\3 == 2 + 1\
 
    \verb\3 == 1.5 * 2\

    \verb\3 == 0.6 * 5\

    \verb\3 != "three"\

    \verb\True and (1 == 3 - 2)\

    \verb\some_int == (some_int / 3) * 3 + some_int % 3\

    \verb\  # Warning: Python 3 requires some_int // 3\

\end{frame}

\section{Functions}

\begin{frame}[fragile]{Functions}
 
\begin{verbatim}
def list_m_powers(x, m):
  n = 1
  powers = []
  while n < m:
    powers.append(x**n)
    n += 1
  return powers
\end{verbatim}
\pause
\begin{verbatim}
simple = lambda x, m: [x**n for n in range(1, m)]

  # list_m_powers(2, 5) == simple(2, 5) 
  # == [2, 4, 8, 16, 32]
\end{verbatim}

\end{frame}

\section{Loops, Conditionals}

\begin{frame}[fragile]{Loops, Conditionals}
 
\begin{verbatim}
def do_stuff(items):
  copy = items[:]
  while copy != []:
    item = copy.pop()
    for index, numb in enumerate(item):
      print numb**index
    if len(item) < 3:
      print "That was easy!\n"
    else:
      print "Whew, now we're done!\n"
\end{verbatim}

\end{frame}

\begin{frame}[fragile]

\begin{verbatim}
# do_stuff([[1, 2], [3, 5, 7, -4]]) returns...
1
5
49
-64
Whew, now we're done!

1
2
That was easy!
\end{verbatim}

\end{frame}


\begin{frame}
That's all folks.
\end{frame}


\end{document}


